\documentclass{article}

\usepackage{amsmath}
\usepackage{bbold}

\begin{document}

\section*{Idea}
The Idea behind the ADT of a Stack is to provide a collection of data with two primary operations:

\begin{itemize}
	\item \emph{push}: Push a new item on top of the stack.
	\item \emph{pop}: Take the item on top of the stack off.
\end{itemize}


\section*{Definition}

We define the ADT as the following 5-Tuple:
\[\mathcal{D} = (N, P, Fs, Ts, Ax),\]
where the components are defined as follows:

\begin{enumerate}
 \item \(N :=\) Stack
 \item \(P := \{ \text{Element} \}\)
 \item \(Fs := \{ \text{stack, push, pop, top, isEmpty} \}\)
 \item \(Ts\) is the set containing the following type specifications:
	\begin{enumerate}
		\item \( \text{stack} : \text{Stack} \)
		\item push : Stack \(\times\) Element \(\rightarrow\) Stack
		\item pop : Stack \(\rightarrow\) Stack \(\cup \{\Omega\}\)
		\item top : Stack \(\rightarrow\) Stack\(\cup \{\Omega\}\)
		\item isEmpty : Stack \(\rightarrow \mathbb{B}\)
	\end{enumerate}
 \item \(Ax\) is the set containing the following axioms:
	\begin{enumerate}
		\item stack().top() = \(\Omega\)
  \item stack().pop() = \(\Omega\)
  \item \(S\).push(\(x\)).top() = \(x\)
  \item \(S\).push(\(x\)).pop() = \(S\)
  \item \(S\).top() = \(\Omega \iff\) S.isEmpty() = true
	\end{enumerate}
\end{enumerate}


\section*{Implementation}
TBD

\end{document}