\documentclass{article}

\usepackage{amsmath}
\usepackage{bbold}

\begin{document}

\section*{Idea}
The Idea behind the ADT of a Queue is to provide a collection of data with two primary operations:

\begin{itemize}
	\item \emph{push}: Push a new item on top of the Queue.
	\item \emph{pop}: Take the item on top of the Queue off.
\end{itemize}


\section*{Definition}

We define the ADT as the following 5-Tuple:
\[\mathcal{D} = (N, P, Fs, Ts, Ax),\]
where the components are defined as follows:

\begin{enumerate}
 \item \(N :=\) Queue
 \item \(P := \{ \text{Element} \}\)
 \item \(Fs := \{ \text{queue, enqueue, dequeue, peek, length} \}\)
 \item \(Ts\) is the set containing the following type specifications:
	\begin{enumerate}
		\item \( \text{queue} : \text{Queue} \)
  		\item length : Queue \(\rightarrow \mathbb{N_0}\)
		\item enqueue : Queue \(\times\) Element \(\rightarrow\) Queue
		\item dequeue : Queue \(\rightarrow\) Queue \(\cup\) \(\{\Omega\}\)
		\item peek : Queue \(\rightarrow\) Element \(\cup\) \(\{\Omega\}\)
	\end{enumerate}
 \item \(Ax\) is the set containing the following axioms. \\
	\(\forall Q \in \text{Queue} : x,y \in \text{Element} :\)
	\begin{enumerate}
  \item queue().peek() = \(\Omega\)
  \item queue().length() = 0
  \item \(Q\).enqueue(\(x\)).length() = \(Q\).length() + 1
  \item \(Q\).length() \(>\) 0 \(\rightarrow\) \(Q\).dequeue().length() = \(Q\).length() - 1
  \item \(Q\).length() \(> 0 \rightarrow Q\).enqueue(\(x\)).dequeue() = \(Q\).dequeue().enqueue(\(x\))
  \item queue().enqueue(\(x\)).dequeue() = queue()
  \item queue().dequeue() = \(\Omega\)
  \item \(Q\).length() \(> 0 \rightarrow Q\).enqueue(\(x\)).peek() = \(Q\).peek()
  \item \(Q\).length() \(= 0 \rightarrow Q\).enqueue(\(x\)).peek() = \(x\)
	\end{enumerate}
\end{enumerate}


\section*{Implementation}
TBD

\end{document}