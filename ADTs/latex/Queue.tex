\documentclass{article}

\usepackage[english]{babel}
\usepackage{amsmath}
\usepackage{amsthm}
\usepackage{bbold}

\newtheorem{theorem}{Theorem}
\newtheorem{corollary}{Corollary}

% pparagraph is for Paragraph headings in Proofs or similar sections
\newcommand{\pparagraph}[1]{\textbf{\underline{#1}}}

\begin{document}

\section*{Idea}
The Queue ADT follows the "First-In, First-Out" Principle, which means the element that was first added is also first removed again. It has similar methods to the Stack, with a different removing-function.


\section*{Definition}

We define the ADT as the following 5-Tuple:
\[\mathcal{D} = (N, P, Fs, Ts, Ax),\]
where the components are defined as follows:

\begin{enumerate}
 \item \(N :=\) Queue
 \item \(P := \{ \text{Element} \}\)
 \item \(Fs := \{ \text{queue, enqueue, dequeue, peek, length} \}\)
 \item \(Ts\) is the set containing the following type specifications:
	\begin{enumerate}
		\item \( \text{queue} : \text{Queue} \)
  		\item length : Queue \(\rightarrow \mathbb{N_0}\)
		\item enqueue : Queue \(\times\) Element \(\rightarrow\) Queue
		\item dequeue : Queue \(\rightarrow\) Queue \(\cup\) \(\{\Omega\}\)
		\item peek : Queue \(\rightarrow\) Element \(\cup\) \(\{\Omega\}\)
	\end{enumerate}
 \item \(Ax\) is the set containing the following axioms. \\
	\(\forall Q \in \text{Queue} : x,y \in \text{Element} :\)
	\begin{enumerate}
  \item queue().peek() = \(\Omega\)
  \item queue().length() = 0
  \item \(Q\).enqueue(\(x\)).length() = \(Q\).length() + 1
  \item \(Q\).length() \(>\) 0 \(\rightarrow\) \(Q\).dequeue().length() = \(Q\).length() - 1
  \item \(Q\).length() \(> 0 \rightarrow Q\).enqueue(\(x\)).dequeue() = \(Q\).dequeue().enqueue(\(x\))
  \item queue().enqueue(\(x\)).dequeue() = queue()
  \item queue().dequeue() = \(\Omega\)
  \item \(Q\).length() \(> 0 \rightarrow Q\).enqueue(\(x\)).peek() = \(Q\).peek()
  \item \(Q\).length() \(= 0 \rightarrow Q\).enqueue(\(x\)).peek() = \(x\)
	\end{enumerate}
\end{enumerate}


\section*{Theorems}
In this section, the definition from above is used to prove useful theorems about the Queue ADT.

\begin{corollary} \label{theorem_length_equiv_enqueue_amount}
	\(Q\).length() is equals to the amount of function calls of "enqueue" minus the amount of function calls of "dequeue"
\end{corollary}

\begin{theorem}
	Dequeuing an element from a Queue \(Q\) removes the element obtained through \(Q\).peek().
\end{theorem}
\begin{proof}
	Let \(x = Q\).peek() and \(x \in\) Element. Note, that this means, that \(Q\).length() \(> 0\), since \(x\) would not be an Element otherwise. \\
	We will prove the theorem via induction over \(n\), where \(n = Q\).length() now:
	
	\pparagraph{Base Case:} \(n = 1\)
	\begin{align*}
		Q_1 &= queue().enqueue(x) &| \text{ Corollary \ref{theorem_length_equiv_enqueue_amount}} \\
		Q_1.peek() &= queue().enqueue(x).peek() \\
		Q_1.peek() &= x
	\end{align*}
	
	\pparagraph{Induction Step:} \(n \rightarrow n+1\)
	\begin{align*}
		Q_{n+1} &= Q_n.enqueue(y) \\
		Q_{n+1} &= queue().enqueue(x_1)....enqueue(x_n).enqueue(y) \\
		&| \text{ Corollary \ref{theorem_length_equiv_enqueue_amount}} \\
		Q_{n+1}.peek() &= queue().enqueue(x_1)....enqueue(x_n).enqueue(y).peek() \\
		Q_{n+1}.peek() &= queue().enqueue(x_1)....enqueue(x_n).peek() \\
		&| \text{ Axiom (h)} \\
		Q_{n+1}.peek() &= queue().enqueue(x_1).peek() \\
		&| \text{ Axiom (h) applied n-1 times} \\
		Q_{n+1}.peek() &= x_1 \\
		&| \text{ Axiom (i)} \\
		Q_{n+1}.peek() &= queue().enqueue(x_1).peek() \\
		Q_{n+1}.peek() &= Q_1.peek()
	\end{align*}
\end{proof}
	
	
\begin{theorem}
	Queues follow the "First-In, First-Out"-Principle. That means, that the 1st,2nd,...,\(n\)th element to enqueued is also going to be the 1st,2nd,...,\(n\)th element to be dequeued.
\end{theorem}
\begin{proof}
	Let \(Q_n\) be a Queue, such that \(Q\).length() \(= n\). It follows then from Corollary \ref{theorem_length_equiv_enqueue_amount}, that
	\[Q_n = queue().enqueue(x_1)....enqueue(x_n)\]
	holds. Further, we know from Axiom (e), that the following holds for all \(n > 0\):
	\begin{align*}
		Q_n.dequeue() &= queue().enqueue(x_1).enqueue(x_2)....enqueue(x_n).dequeue() \\
		&= queue().enqueue(x_1).enqueue(x_2)....dequeue().enqueue(x_n) \\
		&= queue().enqueue(x_1).enqueue(x_2).dequeue()....enqueue(x_n) \\
		&= queue().enqueue(x_1).dequeue().enqueue(x_2)....enqueue(x_n)
	\end{align*}
	From Axiom (f), that the above is equivalent with the following:
	\begin{align*}
		Q_n.dequeue() &= queue().enqueue(x_2)....enqueue(x_n)
	\end{align*}
	It can easily be proved (per induction for example), that by repeating the same steps \(n\) times this holds for all elements aside from \(x_1\) as well, yet that proof is left out here for brevity.
\end{proof}
	


\section*{Implementation}
TBD

\end{document}