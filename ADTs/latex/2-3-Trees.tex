\documentclass{article}

\usepackage{amsmath}
\usepackage{bbold}

\begin{document}

\section*{Idea}
...

\section*{Definition}
We define \(T\), the set of 2-3-Trees, inductively:

\begin{enumerate}
	\item \(Nil \in T\)
 \item \(Two(l, k, r) \in T\) if and only if:
	\begin{enumerate}
		\item \(l, r \in T\)
  		\item \(k \in Keys\)
    \item \(l < k < r\)
    \item \(l.height() = m.height()\)
	\end{enumerate}
 \item \(Three(l, k_1, m, k_2, r) \in T\) if and only if:
	\begin{enumerate}
		\item \(l, m, r \in T\)
		\item \(k_1, k_2 \in Keys\)
  \item \(l < k_1 < m < k_2 < r\)
  \item \(l.height() = m.height() = r.height()\)
	\end{enumerate}
\end{enumerate}

There are also 4-Nodes:
\[Four(l, k_l, m_l, k_m, m_r, k_r, r)\]
is a 4-Node iff:
\begin{enumerate}
	\item \(l, m_l, m_r, r \in T\)
 \item \(k_l, k_m, k_r \in Keys\)
 \item \(l < k_l < m_l < k_m < m_r < k_r < r\)
 \item \(l.height() = m_l.height() = m_r.height() = r.height()\)
\end{enumerate}


We define \(insert: T \times Keys \rightarrow T\) using 3 auxilliary methods:
\begin{enumerate}
	\item \(ins: T \times Keys \rightarrow T^*\)
		\(t.ins(k)\) may not contain more than one 4-Node, which has to be directly below the root-node - or if \(t\) consists of only one node, then the 4-node may be at the root too.
 \item \(restore: T^* \rightarrow T^*\)
		\(restore\) moves the 4-node up to the root.
 \item \(grow: T^* \rightarrow T\)
\end{enumerate}

\section*{Implementation}
TBD


\end{document}