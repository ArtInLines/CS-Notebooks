\documentclass{article}

\usepackage{amsmath}
\usepackage{bbold}

\begin{document}

\section*{Idea}
The idea is to map specific keys to values. A map must thus be able to retrieve the value associated with some key, insert/overwrite the value associated with some key and delete keys.

There are several abstract data types that implement this ADT with further specifications.

\section*{Definition}
We define the ADT as the following 5-Tuple:
\[\mathcal{D} = (N, P, Fs, Ts, Ax),\]
where the components are defined as follows:

\begin{enumerate}
 \item \(N :=\) Map
 \item \(P := \{ \text{Key, Value} \}\)
 \item \(Fs := \{ \text{map, find, insert, delete} \}\)
 \item \(Ts\) is the set containing the following type specifications:
	\begin{enumerate}
		\item \( \text{map} : \text{Map} \)
		\item find : Map \(\times\) Key \(\rightarrow\) Value \(\cup \{\Omega\}\)
		\item insert : Map \(\times\) Key \(\times\) Value \(\rightarrow\) Map \(\cup \{\Omega\}\)
		\item find : Map \(\times\) Key \(\rightarrow\) Map
	\end{enumerate}
 \item \(Ax\) is the set containing the following axioms. \\
	\(\forall L \in \text{List} : x \in \text{Element} : i \in \mathbb{N_0} :\)
	\begin{enumerate}
		\item map().find(\(k\)) = \(\Omega\)
  \item \(m\).insert(\(k, v\)).find(\(k\)) = v
  \item \(k_1 \ne k_2 \rightarrow\) \(m\).insert(\(k_1, v\)).find(\(k_2\)) = \(m\).find(\(k_2\))
  \item \(m\).delete(\(k\)).find(\(k\)) = \(\Omega\)
  \item \(k_1 \ne k_2 \rightarrow m\).delete(\(k_1\)).find(\(k_2\)) = \(m\).find(\(k_2\))
	\end{enumerate}
\end{enumerate}


\section*{Implementation}
TBD


\end{document}